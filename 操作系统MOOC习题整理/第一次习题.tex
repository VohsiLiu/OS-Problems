\subsection*{第一次习题}
\setcounter{problemname}{0}

\begin{problem}
	操作系统是对 \myline 进行管理的软件。
	\textbf{B}
	\vspace{-0.5em}
	\begin{multicols}{4}
		\begin{enumerate}[label=\Alph*.]
			\item 软件
			\item 计算机资源
			\item 硬件
			\item 应用程序
		\end{enumerate}
	\end{multicols}
	\vspace{-1em}
\end{problem}


\begin{problem}
	配置了操作系统的机器是一台比原来的物理机器功能更强的计算机,这样的计算机只是一台逻辑上的计算机,称为 \myline 计算机。
	\textbf{B}
	\vspace{-0.5em}
	\begin{multicols}{4}
		\begin{enumerate}[label=\Alph*.]
			\item 并行
			\item 虚拟
			\item 共享
			\item 真实
		\end{enumerate}
	\end{multicols}
	\vspace{-1em}
\end{problem}


\begin{problem}
	\myline 不是一个操作系统环境。
	\textbf{C}
	\vspace{-0.5em}
	\begin{multicols}{4}
		\begin{enumerate}[label=\Alph*.]
			\item Windows CE
			\item Solaris
			\item Celeron
			\item Linuxs
		\end{enumerate}
	\end{multicols}
	\vspace{-1em}
\end{problem}


\begin{problem}
	\myline 该操作系统的系统响应时间的重要性超过协同资源的利用率,它被广泛地应用于卫星控制、导弹发射、工业控制、飞机订票业务等领域。
	\textbf{C}
	\vspace{-0.5em}
	\begin{multicols}{4}
		\begin{enumerate}[label=\Alph*.]
			\item 多用户操作系统
			\item 分时操作系统
			\item 实时操作系统
			\item 批处理操作系统			
		\end{enumerate}
	\end{multicols}
	\vspace{-1em}
\end{problem}


\begin{problem}
	允许在一台主机上同时连接多个终端,各个用户可以通过各自的终端交互使用计算机,这样的操作系统是 \myline 。
	\textbf{D}
	\vspace{-0.5em}
	\begin{multicols}{4}
		\begin{enumerate}[label=\Alph*.]
			\item 网络操作系统
			\item 分布式操作系统
			\item 批处理操作系统
			\item 分时操作系统
		\end{enumerate}
	\end{multicols}
	\vspace{-1em}
\end{problem}


\begin{problem}
	如果分时系统的时间片一定,那么 \myline,则响应时间越长。
	\textbf{B}
	\vspace{-0.5em}
	\begin{multicols}{4}
		\begin{enumerate}[label=\Alph*.]
			\item 内存越多
			\item 用户数越多
			\item 内存越少
			\item 用户数越少
		\end{enumerate}
	\end{multicols}
	\vspace{-1em}
\end{problem}


\begin{problem}
	系统调用是 \myline。
	\textbf{B}
	\vspace{-0.5em}
	\begin{multicols}{2}
		\begin{enumerate}[label=\Alph*.]
			\item 用户编写的一个子程序
			\item 操作系统向用户程序提供的接口
			\item 高级语言中的库程序
			\item 操作系统中的一条命令
		\end{enumerate}
	\end{multicols}
	\vspace{-1em}
\end{problem}


\begin{problem}
	‍实时操作系统必须在 \myline 内处理来自外部的事件。
	\textbf{A}
	\vspace{-0.5em}
	\begin{multicols}{4}
		\begin{enumerate}[label=\Alph*.]
			\item 规定时间
			\item 周转时间
			\item 响应时间
			\item 调度时间
		\end{enumerate}
	\end{multicols}
	\vspace{-1em}
\end{problem}


\begin{problem}
	‍‍实时系统 \myline 。
	\textbf{D}
	%\vspace{-0.5em}
	%\begin{multicols}{1}
		\begin{enumerate}[label=\Alph*.]
			\item 强调系统资源的利用率
			\item 是依赖人为干预的监督和控制系统
			\item 实质上是批处理系统和分时系统的结合
			\item 必须既要及时响应、快速处理,又要有高可靠性和安全性
		\end{enumerate}
	%\end{multicols}
	%\vspace{-1em}
\end{problem}


\begin{problem}
	‍用户程序的输入和输出操作实际上由 \myline 完成。
	\textbf{A}
	\vspace{-0.5em}
	\begin{multicols}{4}
		\begin{enumerate}[label=\Alph*.]
			\item 操作系统
			\item 标准库程序
			\item 编译系统
			\item 程序设计语言
		\end{enumerate}
	\end{multicols}
	\vspace{-1em}
\end{problem}


\begin{problem}
	在操作系统中,并发性是指 \myline。
	\textbf{D}
	\vspace{-0.5em}
	\begin{multicols}{2}
		\begin{enumerate}[label=\Alph*.]
			\item 若干个时间在不同的时间间隔内发生
			\item 若干个时间在不同时刻发生
			\item 若干个事件在同一时刻发生
			\item 若干个事件在同一时间间隔内发生
		\end{enumerate}
	\end{multicols}
	\vspace{-1em}
\end{problem}


\begin{problem}
	若把操作系统看成计算机系统资源的管理者,下面的 \myline 不属于操作系统所管理的资源。
	\textbf{C}
	\vspace{-0.5em}
	\begin{multicols}{4}
		\begin{enumerate}[label=\Alph*.]
			\item 程序
			\item CPU
			\item 中断
			\item 主存
		\end{enumerate}
	\end{multicols}
	\vspace{-1em}
\end{problem}



\begin{problem}
	多道程序设计是指 \myline。   
	\textbf{C}
	\vspace{-0.5em}
	\begin{multicols}{2}
		\begin{enumerate}[label=\Alph*.]
			\item 在分布系统中同一时刻运行多个程序
			\item 在实时系统中并发运行多个程序
			\item 在一台处理机上并发运行多个程序
			\item 在一台处理机上同一时刻运行多个程序
		\end{enumerate}
	\end{multicols}
	\vspace{-1em}
\end{problem}



\begin{problem}
	提高处理器资源利用率的关键技术是 \myline。
	\textbf{A}
	\vspace{-0.5em}
	\begin{multicols}{4}
		\begin{enumerate}[label=\Alph*.]
			\item 多道程序设计技术
			\item 交换技术
			\item SPOOLing技术
			\item 虚拟技术
		\end{enumerate}
	\end{multicols}
	\vspace{-1em}
\end{problem}



\begin{problem}
	操作系统中采用多道程序设计提高CPU和外部设备的 \myline。
	\textbf{C}
	\vspace{-0.5em}
	\begin{multicols}{4}
		\begin{enumerate}[label=\Alph*.]
			\item 可靠性
			\item 稳定性
			\item 利用率
			\item 兼容性
		\end{enumerate}
	\end{multicols}
	\vspace{-1em}
\end{problem}


\begin{problem}
	引入多道程序设计技术的前提条件之一是系统具有 \myline。
	\textbf{C}
	\vspace{-0.5em}
	\begin{multicols}{4}
		\begin{enumerate}[label=\Alph*.]
			\item 多个CPU
			\item 多个终端
			\item 中断功能
			\item 分时功能
		\end{enumerate}
	\end{multicols}
	\vspace{-1em}
\end{problem}



\begin{problem}
	当计算机提供了管态和目态时,\myline 必须在管态下执行。
	\textbf{B}
	\vspace{-0.5em}
	\begin{multicols}{2}
		\begin{enumerate}[label=\Alph*.]
			\item 把运算结果送入内存的指令
			\item 输入/输出指令
			\item 算术运算指令
			\item 从内存取数的指令
		\end{enumerate}
	\end{multicols}
	\vspace{-1em}
\end{problem}



\begin{problem}
    当CPU执行操作系统代码时,称处理机处于 \myline。
	\textbf{B}
	\vspace{-0.5em}
	\begin{multicols}{4}
		\begin{enumerate}[label=\Alph*.]
			\item 目态
			\item 管态
			\item 自由态
			\item 就绪态
		\end{enumerate}
	\end{multicols}
	\vspace{-1em}
\end{problem}



\begin{problem}
	特权指令是指 \myline。
	\textbf{C}
	\vspace{-0.5em}
	\begin{multicols}{2}
		\begin{enumerate}[label=\Alph*.]
			\item 控制指令
			\item 系统管理员可用的指令;
			\item 其执行可能有损系统的安全性
			\item 机器指令
		\end{enumerate}
	\end{multicols}
	\vspace{-1em}
\end{problem}


\begin{problem}
	计算机系统中判断是否有中断事件发生应该在 \myline。
	\textbf{D}
	\vspace{-0.5em}
	\begin{multicols}{2}
		\begin{enumerate}[label=\Alph*.]
			\item 执行P操作后
			\item 由用户态转入核心态时
			\item 若干个事件在同一时刻发生
			\item 执行完一条指令后
		\end{enumerate}
	\end{multicols}
	\vspace{-1em}
\end{problem}