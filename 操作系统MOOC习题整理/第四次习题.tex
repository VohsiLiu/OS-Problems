\subsection*{第四次习题}
\setcounter{problemname}{0}

\begin{problem}
	按 \myline 分类可将设备分为块设备和字符设备。
	\textbf{D}
	\vspace{-0.5em}
	\begin{multicols}{4}
		\begin{enumerate}[label=\Alph*.]
			\item 共享属性
			\item 操作特性
			\item 从属关系
			\item 信息交换单位
		\end{enumerate}
	\end{multicols}
	\vspace{-1em}
\end{problem}


\begin{problem}
	CPU输出数据的速度远远高于打印机的打印速度,为了解决这一矛盾,可采用 \myline。
	\textbf{A}
	\vspace{-0.5em}
	\begin{multicols}{4}
		\begin{enumerate}[label=\Alph*.]
			\item 缓冲技术
			\item 覆盖技术
			\item 虚存技术
			\item 并行技术
		\end{enumerate}
	\end{multicols}
	\vspace{-1em}
\end{problem}


\begin{problem}
	通过硬件和软件的功能扩充,把原来独占的设备改造成能为若干用户共享的设备,这种设备称为 \myline。
	\textbf{D}
	\vspace{-0.5em}
	\begin{multicols}{4}
		\begin{enumerate}[label=\Alph*.]
			\item 系统设备
			\item 存储设备
			\item 用户设备
			\item 虚拟设备
		\end{enumerate}
	\end{multicols}
	\vspace{-1em}
\end{problem}


\begin{problem}
	通道又称I/O处理机,它用于实现 \myline 之间的信息传输。
	\textbf{A}
	\vspace{-0.5em}
	\begin{multicols}{4}
		\begin{enumerate}[label=\Alph*.]
			\item 内存与外设
			\item 内存与外存
			\item CPU与外设
			\item CPU与外存
		\end{enumerate}
	\end{multicols}
	\vspace{-1em}
\end{problem}


\begin{problem}
	为了使多个进程能有效地同时处理输入和输出,最好使用 \myline 结构的缓冲技术。
	\textbf{B}
	\vspace{-0.5em}
	\begin{multicols}{4}
		\begin{enumerate}[label=\Alph*.]
			\item 单缓冲
			\item 缓冲池
			\item 双缓冲
			\item 循环缓冲
		\end{enumerate}
	\end{multicols}
	\vspace{-1em}
\end{problem}


\begin{problem}
	如果I/O设备与存储设备进行数据交换不经过CPU来完成,这种数据交换方式是 \myline。
	\textbf{A}
	\vspace{-0.5em}
	\begin{multicols}{4}
		\begin{enumerate}[label=\Alph*.]
			\item DMA方式
			\item 中断方式
			\item 无条件存取方式
			\item 程序轮询
		\end{enumerate}
	\end{multicols}
	\vspace{-1em}
\end{problem}


\begin{problem}
	​在中断处理中,输入/输出中断可能是指 \myline:\ding{192}设备出错,\ding{193}数据传输结束。
	\textbf{D}
	\vspace{-0.5em}
	\begin{multicols}{4}
		\begin{enumerate}[label=\Alph*.]
			\item \ding{193}
			\item \ding{192}
			\item 都不是
			\item \ding{192}和\ding{193}
		\end{enumerate}
	\end{multicols}
	\vspace{-1em}
\end{problem}


\begin{problem}
	在采用SPOOLing技术的系统中,用户的打印结果首先被送到 \myline。
	\textbf{D}
	\vspace{-0.5em}
	\begin{multicols}{4}
		\begin{enumerate}[label=\Alph*.]
			\item 打印机
			\item 终端
			\item 内存固定区域
			\item 磁盘固定区域
		\end{enumerate}
	\end{multicols}
	\vspace{-1em}
\end{problem}


\begin{problem}
	大多数低速设备都属于 \myline 设备。
	\textbf{C}
	\vspace{-0.5em}
	\begin{multicols}{4}
		\begin{enumerate}[label=\Alph*.]
			\item 虚拟
			\item SPOOLing
			\item 独享
			\item 共享
		\end{enumerate}
	\end{multicols}
	\vspace{-1em}
\end{problem}


\begin{problem}
	\myline 是直接存取的存储设备。
	\textbf{B}
	\vspace{-0.5em}
	\begin{multicols}{4}
		\begin{enumerate}[label=\Alph*.]
			\item 键盘显示终端
			\item 磁盘
			\item 磁带
			\item 打印机
		\end{enumerate}
	\end{multicols}
	\vspace{-1em}
\end{problem}


\begin{problem}
	操作系统中的SPOOLing技术,实质是指 \myline 转化为共享设备的技术。
	\textbf{A}
	\vspace{-0.5em}
	\begin{multicols}{4}
		\begin{enumerate}[label=\Alph*.]
			\item 独占设备
			\item 脱机设备
			\item 块设备
			\item 虚拟设备
		\end{enumerate}
	\end{multicols}
	\vspace{-1em}
\end{problem}


\begin{problem}
	在操作系统中,\myline 指的是一种硬件机制。
	\textbf{C}
	\vspace{-0.5em}
	\begin{multicols}{4}
		\begin{enumerate}[label=\Alph*.]
			\item SPOOLing技术
			\item 缓冲池
			\item 通道技术
			\item 内存覆盖技术
		\end{enumerate}
	\end{multicols}
	\vspace{-1em}
\end{problem}


\begin{problem}
	在操作系统中,用户程序申请使用I/O设备时,通常采用 \myline。
	\textbf{C}
	\vspace{-0.5em}
	\begin{multicols}{4}
		\begin{enumerate}[label=\Alph*.]
			\item 独占设备名
			\item 虚拟设备名
			\item 逻辑设备名
			\item 物理设备名
		\end{enumerate}
	\end{multicols}
	\vspace{-1em}
\end{problem}


\begin{problem}
	采用假脱机技术,将磁盘的一部分作为公共缓冲区以代替打印机,用户对打印机的操作实际上是对磁盘的存储操作,用以代替打印机的部分是 \myline。
	\textbf{B}
	\vspace{-0.5em}
	\begin{multicols}{4}
		\begin{enumerate}[label=\Alph*.]
			\item 独占设备
			\item 虚拟设备
			\item 一般物理设备
			\item 共享设备
		\end{enumerate}
	\end{multicols}
	\vspace{-1em}
\end{problem}


\begin{problem}
	\myline 算法是设备分配常用的一种算法。
	\textbf{C}
	\vspace{-0.5em}
	\begin{multicols}{4}
		\begin{enumerate}[label=\Alph*.]
			\item 首次适应
			\item 最佳适应
			\item 先来先服务
			\item 短作业优先
		\end{enumerate}
	\end{multicols}
	\vspace{-1em}
\end{problem}


\begin{problem}
	‌将系统中的每一台设备按某种原则进行统一的编号,这些编号作为区分硬件和识别设备的代号,该编号称为设备的 \myline。
	\textbf{D}
	\vspace{-0.5em}
	\begin{multicols}{4}
		\begin{enumerate}[label=\Alph*.]
			\item 符号名
			\item 相对号
			\item 类型号
			\item 绝对号
		\end{enumerate}
	\end{multicols}
	\vspace{-1em}
\end{problem}


\begin{problem}
	通道程序是 \myline。
	\textbf{C}
	\vspace{-0.5em}
	\begin{multicols}{2}
		\begin{enumerate}[label=\Alph*.]
			\item 可以由高级语言编写
			\item 由一系列机器指令组成
			\item 由一系列通道指令组成
			\item 就是通道控制器
		\end{enumerate}
	\end{multicols}
	\vspace{-1em}
\end{problem}


\begin{problem}
	I/O软件的分层结构中,\myline 负责将把用户提交的逻辑I/O请求转化为物理I/O操作的启动和执行。
	\textbf{B}
	\vspace{-0.5em}
	\begin{multicols}{2}
		\begin{enumerate}[label=\Alph*.]
			\item 独立于设备的I/O软件
			\item 设备驱动程序
			\item 用户空间的I/O软件
			\item I/O中断处理程序
		\end{enumerate}
	\end{multicols}
	\vspace{-1em}
\end{problem}


\begin{problem}
	使用SPOOLing系统的目的是为了提高 \myline 的使用效率。
	\textbf{B}
	\vspace{-0.5em}
	\begin{multicols}{4}
		\begin{enumerate}[label=\Alph*.]
			\item 操作系统
			\item I/O设备
			\item 内存
			\item CPU
		\end{enumerate}
	\end{multicols}
	\vspace{-1em}
\end{problem}


\begin{problem}
	下列算法中,用于磁盘移臂调度的是 \myline。
	\textbf{D}
	\vspace{-0.5em}
	\begin{multicols}{2}
		\begin{enumerate}[label=\Alph*.]
			\item 时间片轮转法
			\item 优先级高者优先算法
			\item LRU算法
			\item 最短寻找时间优先算法
		\end{enumerate}
	\end{multicols}
	\vspace{-1em}
\end{problem}
