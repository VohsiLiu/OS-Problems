\subsection*{第二次习题}
\setcounter{problemname}{0}

\begin{problem}
	\myline 优先权是在创建进程时确定的,确定之后在整个进程运行期间不再改变。
	\textbf{B}
	\vspace{-0.5em}
	\begin{multicols}{4}
		\begin{enumerate}[label=\Alph*.]
			\item 动态
			\item 静态
			\item 短作业
			\item 先来先服务
		\end{enumerate}
	\end{multicols}
	\vspace{-1em}
\end{problem}


\begin{problem}
	下列进程状态变化中,\myline 变化是不可能发生的。
	\textbf{C}
	\vspace{-0.5em}
	\begin{multicols}{4}
		\begin{enumerate}[label=\Alph*.]
			\item 等待$\rightarrow$就绪
			\item 运行$\rightarrow$就绪
			\item 等待$\rightarrow$运行
			\item 运行$\rightarrow$等待
		\end{enumerate}
	\end{multicols}
	\vspace{-1em}
\end{problem}


\begin{problem}
	​当 \myline 时,进程从运行状态变为就绪状态。
	\textbf{D}
	\vspace{-0.5em}
	\begin{multicols}{2}
		\begin{enumerate}[label=\Alph*.]
			\item 进程被调度程序选中
			\item 等待某一事件
			\item 等待的事件发生
			\item 时间片到
		\end{enumerate}
	\end{multicols}
	\vspace{-1em}
\end{problem}


\begin{problem}
	进程管理中,当 \myline 时进程从阻塞态变成就绪态。
	\textbf{D}
	\vspace{-0.5em}
	\begin{multicols}{2}
		\begin{enumerate}[label=\Alph*.]
			\item 进程被进程调度程序选中
			\item 时间片用完
			\item 等待一个事件
			\item 等待的事件发生
		\end{enumerate}
	\end{multicols}
	\vspace{-1em}
\end{problem}


\begin{problem}
	下面对进程的描述中,错误的是 \myline。
	\textbf{A}
	\vspace{-0.5em}
	\begin{multicols}{2}
		\begin{enumerate}[label=\Alph*.]
			\item 进程是指令的集合
			\item 进程执行需要处理机
			\item 进程是有生命周期的
			\item 进程是动态的概念
		\end{enumerate}
	\end{multicols}
	\vspace{-1em}
\end{problem}


\begin{problem}
	下面所述步骤中,\myline 不是创建进程所必需的。
	\textbf{B}
	\vspace{-0.5em}
	\begin{multicols}{2}
		\begin{enumerate}[label=\Alph*.]
			\item 建立一个进程控制块
			\item 由调度程序为进程分配CPU
			\item 将进程控制块链入就绪队列
			\item 为进程分配内存
		\end{enumerate}
	\end{multicols}
	\vspace{-1em}
\end{problem}


\begin{problem}
	多道程序环境下,操作系统分配资源以 \myline 为基本单位。
	\textbf{A}
	\vspace{-0.5em}
	\begin{multicols}{4}
		\begin{enumerate}[label=\Alph*.]
			\item 进程
			\item 程序
			\item 线程
			\item 指令
		\end{enumerate}
	\end{multicols}
	\vspace{-1em}
\end{problem}


\begin{problem}
	下述哪一个选项体现了原语的主要特点 \myline。
	\textbf{A}
	\vspace{-0.5em}
	\begin{multicols}{4}
		\begin{enumerate}[label=\Alph*.]
			\item 不可分割性
			\item 共享性
			\item 并发性
			\item 异步性
		\end{enumerate}
	\end{multicols}
	\vspace{-1em}
\end{problem}


\begin{problem}
	‌关于内核级线程,以下描述不正确的是 \myline。
	\textbf{D}
		\begin{enumerate}[label=\Alph*.]
			\item 建立和维护线程的数据结构及保存每个线程的入口
			\item 内核可以将处理器调度直接分配给某个内核级线程
			\item 可以将一个进程的多个线程分派到多个处理器,能够发挥多处理器并行工作的优势
			\item 控制权从一个线程传送到另一个线程时不需要用户态—内核态—用户态的模式切换;
		\end{enumerate}
\end{problem}


\begin{problem}
	一个进程被唤醒意味着 \myline。
	\textbf{B}
	\vspace{-0.5em}
	\begin{multicols}{2}
		\begin{enumerate}[label=\Alph*.]
			\item 该进程重新占有了CPU
			\item 进程变为就绪状态
			\item 其PCB移至等待队列队首
			\item 它的优先权变为最大
		\end{enumerate}
	\end{multicols}
	\vspace{-1em}
\end{problem}


\begin{problem}
	在引入线程的操作系统中,资源分配的基本单位是 \myline。
	\textbf{D}
	\vspace{-0.5em}
	\begin{multicols}{4}
		\begin{enumerate}[label=\Alph*.]
			\item 作业
			\item 线程
			\item 程序
			\item 进程
		\end{enumerate}
	\end{multicols}
	\vspace{-1em}
\end{problem}


\begin{problem}
	在下述关于父进程和子进程的叙述中,正确的是 \myline。
	\textbf{B}
		\begin{enumerate}[label=\Alph*.]
			\item 父进程创建了子进程,因此父进程执行完了,子进程才能运行
			\item 父进程和子进程可以并发执行
			\item 撤销子进程时,应该同时撤销父进程
			\item 撤销父进程时,应该同时撤销子进程
		\end{enumerate}
\end{problem}


\begin{problem}
	对进程的管理和控制使用 \myline。
	\textbf{B}
	\vspace{-0.5em}
	\begin{multicols}{4}
		\begin{enumerate}[label=\Alph*.]
			\item 指令
			\item 原语
			\item 信号量
			\item 信箱通信
		\end{enumerate}
	\end{multicols}
	\vspace{-1em}
\end{problem}


\begin{problem}
	所谓“可重入”程序是指 \myline。
	\textbf{C}
	\vspace{-0.5em}
	\begin{multicols}{2}
		\begin{enumerate}[label=\Alph*.]
			\item 不能够被多个程序同时调用的程序
			\item 在执行过程中其代码自身会发生变化的程序
			\item 能够被多个进程共享的程序
			\item 无限循环程序
		\end{enumerate}
	\end{multicols}
	\vspace{-1em}
\end{problem}


\begin{problem}
	原语是 \myline。
	\textbf{C}
	\vspace{-0.5em}
	\begin{multicols}{2}
		\begin{enumerate}[label=\Alph*.]
			\item 可中断的指令序列
			\item 运行在用户态下的过程
			\item 不可中断的指令序列
			\item 操作系统的内核
		\end{enumerate}
	\end{multicols}
	\vspace{-1em}
\end{problem}


\begin{problem}
	在进程调度算法中,对短进程不利的是 \myline。
	\textbf{D}
	\vspace{-0.5em}
	\begin{multicols}{2}
		\begin{enumerate}[label=\Alph*.]
			\item 短进程优先调度算法
			\item 高响应比优先算法
			\item 多级反馈队列调度算法
			\item 先来先服务算法
		\end{enumerate}
	\end{multicols}
	\vspace{-1em}
\end{problem}


\begin{problem}
	一个可共享的程序在执行过程中是不能被修改的,这样的程序代码应该是 \myline。
	\textbf{B}
	\vspace{-0.5em}
	\begin{multicols}{4}
		\begin{enumerate}[label=\Alph*.]
			\item 封闭的代码
			\item 可重入代码
			\item 可执行代码
			\item 可再现代码
		\end{enumerate}
	\end{multicols}
	\vspace{-1em}
\end{problem}


\begin{problem}
	在进程管理中,当 \myline 时,进程状态从运行态转换到就绪态。
	\textbf{B}
	\vspace{-0.5em}
	\begin{multicols}{2}
		\begin{enumerate}[label=\Alph*.]
			\item 进程被调度程序选中
			\item 时间片用完
			\item 等待某一事件发生
			\item 等待的事件发生
		\end{enumerate}
	\end{multicols}
	\vspace{-1em}
\end{problem}


\begin{problem}
	Solaris的多线程的实现方式为 \myline。
	\textbf{A}
	\vspace{-0.5em}
	\begin{multicols}{4}
		\begin{enumerate}[label=\Alph*.]
			\item 混合式
			\item 纯用户级多线程
			\item 纯内核级线程
			\item 单线程结构进程
		\end{enumerate}
	\end{multicols}
	\vspace{-1em}
\end{problem}


\begin{problem}
	在UNIX系统中运行以下程序,最多可再产生出 \myline 进程?
	\textbf{D}
	\begin{lstlisting}
main( ){
   fork( ); /*←pc(程序计数器),进程A
   fork( );
   fork( );
}
	\end{lstlisting}
	\vspace{-0.5em}
	\begin{multicols}{4}
		\begin{enumerate}[label=\Alph*.]
			\item 3
			\item 9
			\item 5
			\item 7
		\end{enumerate}
	\end{multicols}
	\vspace{-1em}
\end{problem}