\subsection*{第三次习题}
\setcounter{problemname}{0}

\begin{problem}
	静态重定位的时机是 \myline。
	\textbf{B}
	\vspace{-0.5em}
	\begin{multicols}{4}
		\begin{enumerate}[label=\Alph*.]
			\item 程序编译时
			\item 程序装入时
			\item 程序链接时
			\item 程序运行时
		\end{enumerate}
	\end{multicols}
	\vspace{-1em}
\end{problem}


\begin{problem}
	‌能够装入内存任何位置的代码程序必须是\myline。
	\textbf{B}
	\vspace{-0.5em}
	\begin{multicols}{4}
		\begin{enumerate}[label=\Alph*.]
			\item 可定位的
			\item 可动态链接的
			\item 可重入的
			\item 可静态链接的
		\end{enumerate}
	\end{multicols}
	\vspace{-1em}
\end{problem}


\begin{problem}
	在可变式分区管理中,采用内存移动技术的目的是 \myline。
	\textbf{D}
	\vspace{-0.5em}
	\begin{multicols}{4}
		\begin{enumerate}[label=\Alph*.]
			\item 合并分配区
			\item 便于地址转换
			\item 增加主存容量
			\item 合并空闲区
		\end{enumerate}
	\end{multicols}
	\vspace{-1em}
\end{problem}


\begin{problem}
	在存储管理中,采用覆盖与交换技术的目的是 \myline。
	\textbf{B}
	\vspace{-0.5em}
	\begin{multicols}{2}
		\begin{enumerate}[label=\Alph*.]
			\item 物理上扩充主存容量
			\item 减少程序占用的主存空间
			\item 代码在主存中共享
			\item 提高CPU效率
		\end{enumerate}
	\end{multicols}
	\vspace{-1em}
\end{problem}


\begin{problem}
	‍在分区存储管理中,下面的 \myline 最有可能使得高地址空间变成为大的空闲区。
	\textbf{A}
	\vspace{-0.5em}
	\begin{multicols}{4}
		\begin{enumerate}[label=\Alph*.]
			\item 首次适应法
			\item 循环首次适应法
			\item 最佳适应法
			\item 最坏适应法
		\end{enumerate}
	\end{multicols}
	\vspace{-1em}
\end{problem}


\begin{problem}
	以下哪种 \myline 存储管理能提供虚存。
	\textbf{B}
	\vspace{-0.5em}
	\begin{multicols}{4}
		\begin{enumerate}[label=\Alph*.]
			\item 分区方式
			\item 页式
			\item 覆盖
			\item 可重定位分区管理
		\end{enumerate}
	\end{multicols}
	\vspace{-1em}
\end{problem}


\begin{problem}
	在分页式虚存中,分页由 \myline 实现。
	\textbf{B}
	\vspace{-0.5em}
	\begin{multicols}{4}
		\begin{enumerate}[label=\Alph*.]
			\item 程序员
			\item 操作系统
			\item 编译器
			\item 系统调用
		\end{enumerate}
	\end{multicols}
	\vspace{-1em}
\end{problem}


\begin{problem}
	在虚拟页式存储管理方案中,下面 \myline 完成将页面调入内存的工作。
	\textbf{C}
	\vspace{-0.5em}
	\begin{multicols}{4}
		\begin{enumerate}[label=\Alph*.]
			\item 页面淘汰过程
			\item 紧缩技术利用
			\item 缺页中断处理
			\item 工作集模型应用
		\end{enumerate}
	\end{multicols}
	\vspace{-1em}
\end{problem}


\begin{problem}
	采用 \myline 不会产生内部碎片。
	\textbf{A}
	\vspace{-0.5em}
	\begin{multicols}{2}
		\begin{enumerate}[label=\Alph*.]
			\item 分段式存储管理
			\item 固定分区式存储管理
			\item 分页式存储管理
			\item 段页式
		\end{enumerate}
	\end{multicols}
	\vspace{-1em}
\end{problem}



\begin{problem}
	​采用 \myline 存储管理不会产生外部碎片。
	\textbf{C}
	\vspace{-0.5em}
	\begin{multicols}{4}
		\begin{enumerate}[label=\Alph*.]
			\item 分段式
			\item 虚拟分段式
			\item 分页式
			\item 可变分区
		\end{enumerate}
	\end{multicols}
	\vspace{-1em}
\end{problem}


\begin{problem}
	一台机器有48位虚地址和32位物理地址,若页长为8KB,如果设计一个反置页表,则有 \myline 个页表项。
	\textbf{B}
	\vspace{-0.5em}
	\begin{multicols}{4}
		\begin{enumerate}[label=\Alph*.]
			\item $2^{35}$
			\item $2^{19}$
			\item $2^{16}$
			\item $2^{32}$
		\end{enumerate}
	\end{multicols}
	\vspace{-1em}
\end{problem}


\begin{problem}
	作业在执行中发生了缺页中断,经操作系统处理后,应该让其执行 \myline 指令。
	\textbf{C}
	\vspace{-0.5em}
	\begin{multicols}{4}
		\begin{enumerate}[label=\Alph*.]
			\item 被中断的前一条
			\item 被中断的后一条
			\item 被中断的
			\item 启动时的第一条
		\end{enumerate}
	\end{multicols}
	\vspace{-1em}
\end{problem}


\begin{problem}
	在请求分页存储管理中,当访问的页面不在内存时,便产生缺页中断,缺页中断是属于 \myline。
	\textbf{B}
	\vspace{-0.5em}
	\begin{multicols}{4}
		\begin{enumerate}[label=\Alph*.]
			\item 外中断
			\item I/O中断
			\item 程序中断
			\item 访管中断
		\end{enumerate}
	\end{multicols}
	\vspace{-1em}
\end{problem}


\begin{problem}
	通常所说的"存储保护"的基本含义是 \myline。
	\textbf{D}
	\vspace{-0.5em}
	\begin{multicols}{2}
		\begin{enumerate}[label=\Alph*.]
			\item 防止存储器硬件受损
			\item 防止程序被人偷看
			\item 防止程序在内存丢失
			\item 防止程序间相互越界访问
		\end{enumerate}
	\end{multicols}
	\vspace{-1em}
\end{problem}


\begin{problem}
	LRU置换算法所基于的思想是 \myline。
	\textbf{A}
		\begin{enumerate}[label=\Alph*.]
			\item 在最近的过去很久未使用的在最近的将来也不会使用
			\item 在最近的过去用得多的在最近的将来也用得多
			\item 在最近的过去很久未使用的在最近的将来会使用
			\item 在最近的过去用得少的在最近的将来也用得少
		\end{enumerate}
\end{problem}


\begin{problem}
	在下面关于虚拟存储器的叙述中,正确的是 \myline。
	\textbf{D}
		\begin{enumerate}[label=\Alph*.]
			\item 要求程序运行前必须全部装入内存但在运行过程中不必一直驻留在内存
			\item 要求程序运行前不必全部装入内存但是在运行过程中必须一直驻留在内存
			\item 要求程序运行前必须全部装入内存且在运行过程中一直驻留在内存
			\item 要求程序运行前不必全部装入内存且在运行过程中不必一直驻留在内存
		\end{enumerate}
\end{problem}



\begin{problem}
	​虚存的可行性基础是 \myline。
	\textbf{D}
	\vspace{-0.5em}
	\begin{multicols}{4}
		\begin{enumerate}[label=\Alph*.]
			\item 程序执行的离散性
			\item 程序执行的并发性
			\item 程序执行的顺序性
			\item 程序执行的局部性
		\end{enumerate}
	\end{multicols}
	\vspace{-1em}
\end{problem}


\begin{problem}
	把逻辑地址转变为内存的物理地址的过程称作 \myline。
	\textbf{A}
	\vspace{-0.5em}
	\begin{multicols}{4}
		\begin{enumerate}[label=\Alph*.]
			\item 重定位或地址映射
			\item 运行
			\item 编译
			\item 连接
		\end{enumerate}
	\end{multicols}
	\vspace{-1em}
\end{problem}


\begin{problem}
	假定某页式管理系统中,主存128KB,分成32块,块号为$0,1,2,\dots,331$;某作业有5块,其页号为$0,1,2,3,4$,被分别装入主存的$3,8,4,6,9$块中。有一逻辑地址为$[3,70]$。试求出相应的物理地址 \myline (其中方括号中的第一个元素为页号,第二个元素为页内地址,按十进制计算)。
	\textbf{C}
	\vspace{-0.5em}
	\begin{multicols}{4}
		\begin{enumerate}[label=\Alph*.]
			\item 14646
			\item 34576
			\item 24646
			\item 24576
		\end{enumerate}
	\end{multicols}
	\vspace{-1em}
\end{problem}


\begin{problem}
	页面替换算法 \myline 有可能会产生Belady异常现象。
	\textbf{B}
	\vspace{-0.5em}
	\begin{multicols}{4}
		\begin{enumerate}[label=\Alph*.]
			\item Clock
			\item FIFO
			\item OPT
			\item LRU
		\end{enumerate}
	\end{multicols}
	\vspace{-1em}
\end{problem}