\subsection*{第五次习题}
\setcounter{problemname}{0}

\begin{problem}
	Unix系统中,文件的索引结构存放在 \myline 中。
	\textbf{D}
	\vspace{-0.5em}
	\begin{multicols}{4}
		\begin{enumerate}[label=\Alph*.]
			\item 超级块
			\item 空闲块
			\item 目录项
			\item inode节点
		\end{enumerate}
	\end{multicols}
	\vspace{-1em}
\end{problem}


\begin{problem}
	操作系统中对文件进行管理的部分叫做 \myline。
	\textbf{B}
	\vspace{-0.5em}
	\begin{multicols}{4}
		\begin{enumerate}[label=\Alph*.]
			\item 数据库系统
			\item 文件系统
			\item 检索系统
			\item 数据存储系统
		\end{enumerate}
	\end{multicols}
	\vspace{-1em}
\end{problem}


\begin{problem}
	为了解决不同用户文件的“命名冲突”问题,通常在文件系统中采用 \myline。
	\textbf{C}
	\vspace{-0.5em}
	\begin{multicols}{4}
		\begin{enumerate}[label=\Alph*.]
			\item 约定的方法
			\item 索引
			\item 多级目录
			\item 路径
		\end{enumerate}
	\end{multicols}
	\vspace{-1em}
\end{problem}


\begin{problem}
	无结构文件的含义是 \myline。
	\textbf{B}
	\vspace{-0.5em}
	\begin{multicols}{4}
		\begin{enumerate}[label=\Alph*.]
			\item 索引文件
			\item 流式文件
			\item 变长记录的文件
			\item 索引顺序文件
		\end{enumerate}
	\end{multicols}
	\vspace{-1em}
\end{problem}


\begin{problem}
	下列文件中不属于物理文件的是 \myline。
	\textbf{A}
	\vspace{-0.5em}
	\begin{multicols}{4}
		\begin{enumerate}[label=\Alph*.]
			\item 记录式文件
			\item 连续文件
			\item 链接文件
			\item 索引文件
		\end{enumerate}
	\end{multicols}
	\vspace{-1em}
\end{problem}


\begin{problem}
	文件系统的主要目的是 \myline。
	\textbf{B}
	\vspace{-0.5em}
	\begin{multicols}{2}
		\begin{enumerate}[label=\Alph*.]
			\item 实现虚拟存储
			\item 实现对文件的按名存取
			\item 用于存储系统文件
			\item 提高外存的读写速度
		\end{enumerate}
	\end{multicols}
	\vspace{-1em}
\end{problem}


\begin{problem}
	下列文件中属于逻辑结构的文件是 \myline 文件。
	\textbf{A}
	\vspace{-0.5em}
	\begin{multicols}{4}
		\begin{enumerate}[label=\Alph*.]
			\item 流式文件
			\item 库文件
			\item 连续文件
			\item 系统文件
		\end{enumerate}
	\end{multicols}
	\vspace{-1em}
\end{problem}


\begin{problem}
	文件系统采用多级目录结构后,对于不同用户的文件,其文件名 \myline。
	\textbf{A}
	\vspace{-0.5em}
	\begin{multicols}{2}
		\begin{enumerate}[label=\Alph*.]
			\item 可以相同也可以不同
			\item 受系统约束
			\item 应该不同
			\item 应该相同
		\end{enumerate}
	\end{multicols}
	\vspace{-1em}
\end{problem}


\begin{problem}
	文件目录的主要作用是 \myline。
	\textbf{C}
	\vspace{-0.5em}
	\begin{multicols}{4}
		\begin{enumerate}[label=\Alph*.]
			\item 节省空间
			\item 提高外存利用率
			\item 按名存取
			\item 提高速度
		\end{enumerate}
	\end{multicols}
	\vspace{-1em}
\end{problem}


\begin{problem}
	在文件系统中,文件的不同物理结构有不同的优缺点。在下列文件的物理结构中,\myline 具有直接读写文件任意一个记录的能力,又提高了文件存储空间的利用率。
	\textbf{A}
	\vspace{-0.5em}
	\begin{multicols}{4}
		\begin{enumerate}[label=\Alph*.]
			\item 索引结构
			\item Hash结构
			\item 顺序结构
			\item 链接结构
		\end{enumerate}
	\end{multicols}
	\vspace{-1em}
\end{problem}


\begin{problem}
	文件系统用 \myline 组织文件。
	\textbf{B}
	\vspace{-0.5em}
	\begin{multicols}{4}
		\begin{enumerate}[label=\Alph*.]
			\item 堆栈
			\item 目录
			\item 路径
			\item 指针
		\end{enumerate}
	\end{multicols}
	\vspace{-1em}
\end{problem}


\begin{problem}
	文件路径名是指 \myline。
	\textbf{B}
		\begin{enumerate}[label=\Alph*.]
			\item 目录文件名和文件名的集合
			\item 从根目录到文件所经历的路径中的各符号名的集合
			\item 文件名和文件扩展名
			\item 一系列的目录文件名和该文件的文件名
		\end{enumerate}
\end{problem}


\begin{problem}
	一个文件的相对路径名是从 \myline 开始,逐步沿着各级子目录追溯,最后到指定文件的整个通路上所有子目录名组成的一个字符串。
	\textbf{A}
	\vspace{-0.5em}
	\begin{multicols}{4}
		\begin{enumerate}[label=\Alph*.]
			\item 当前目录
			\item 二级目录
			\item 根目录
			\item 多级目录
		\end{enumerate}
	\end{multicols}
	\vspace{-1em}
\end{problem}


\begin{problem}
	对一个文件的访问,常由 \myline 共同限制。
	\textbf{C}
	\vspace{-0.5em}
	\begin{multicols}{2}
		\begin{enumerate}[label=\Alph*.]
			\item 文件属性的口令
			\item 优先级和文件属性
			\item 用户访问权限和文件属性
			\item 用户访问权限和用户优先级
		\end{enumerate}
	\end{multicols}
	\vspace{-1em}
\end{problem}


\begin{problem}
	存放在磁盘上的文件 \myline。
	\textbf{C}
	\vspace{-0.5em}
	\begin{multicols}{2}
		\begin{enumerate}[label=\Alph*.]
			\item 不能随机访问
			\item 只能顺序访问
			\item 既可随机访问,又可顺序访问
			\item 只能随机访问
		\end{enumerate}
	\end{multicols}
	\vspace{-1em}
\end{problem}


\begin{problem}
	在文件系统中,位示图可用于 \myline。
	\textbf{C}
	\vspace{-0.5em}
	\begin{multicols}{2}
		\begin{enumerate}[label=\Alph*.]
			\item 内存空间的共享
			\item 实现文件的保护和保密
			\item 磁盘空间的管理
			\item 文件目录的查找
		\end{enumerate}
	\end{multicols}
	\vspace{-1em}
\end{problem}


\begin{problem}
	常用的文件存取方法有两种:顺序存取和 \myline 存取。
	\textbf{B}
	\vspace{-0.5em}
	\begin{multicols}{4}
		\begin{enumerate}[label=\Alph*.]
			\item 顺序
			\item 随机
			\item 串联
			\item 流式
		\end{enumerate}
	\end{multicols}
	\vspace{-1em}
\end{problem}


\begin{problem}
	Unix系统中,通过 \myline 实现文件系统的按名存取功能。
	\textbf{A}
	\vspace{-0.5em}
	\begin{multicols}{4}
		\begin{enumerate}[label=\Alph*.]
			\item 目录项
			\item 超级块
			\item 空闲块
			\item inode节点
		\end{enumerate}
	\end{multicols}
	\vspace{-1em}
\end{problem}


\begin{problem}
	Unix文件系统中,打开文件的系统调用open输入参数包含 \myline。
	\textbf{A}
	\vspace{-0.5em}
	\begin{multicols}{4}
		\begin{enumerate}[label=\Alph*.]
			\item 文件名
			\item 文件描述符
			\item inode
			\item inode号
		\end{enumerate}
	\end{multicols}
	\vspace{-1em}
\end{problem}


\begin{problem}
	Unix文件系统中,打开文件的系统调用open返回值是 \myline。
	\textbf{A}
	\vspace{-0.5em}
	\begin{multicols}{4}
		\begin{enumerate}[label=\Alph*.]
			\item 文件描述符(字)
			\item inode
			\item 文件名
			\item inode号
		\end{enumerate}
	\end{multicols}
	\vspace{-1em}
\end{problem}
